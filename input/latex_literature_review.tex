\documentclass{article}
\title{A Brief Literature Review on LaTeX}
\author{Example Author}
\date{\today}

\begin{document}
\maketitle

\section{Introduction}
LaTeX has long been a foundational tool for scientific and technical writing,
valued for its precision in mathematical typesetting and its separation of
content from presentation. This short review highlights the recurring themes
in discussions of LaTeX usage across academic and technical communities.

\section{Core Themes in the Literature}
Two consistent themes appear in the literature: quality and control. Authors
emphasize that LaTeX produces stable, high-quality typography and gives writers
fine-grained control over structure and notation. A second theme is portability,
where source files are plain text and easy to version, compare, and reuse.

\section{Adoption in Academic Workflows}
Studies of writing practices note that LaTeX is frequently adopted in fields
with heavy mathematical content, such as physics, computer science, and
engineering. Its extensibility through packages also supports specialized
formats for journals and conferences, easing submission workflows.

\section{Challenges and Learning Curve}
While the benefits are substantial, reports also note the learning curve for
new users. Authors recommend gradual onboarding, templates, and clear
documentation to reduce initial friction.

\section{Conclusion}
Overall, the literature presents LaTeX as a durable, standards-based platform
for technical communication, with trade-offs that can be mitigated through
instruction and tooling.

\end{document}
